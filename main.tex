%%%%%%%%%%%%%%%%%%%%%%%%%%%%%%%%%%%%%%%%%
% Beamer Presentation
% LaTeX Template
% Version 1.0 (10/11/12)
%
% This template has been downloaded from:
% http://www.LaTeXTemplates.com
%
% License:
% CC BY-NC-SA 3.0 (http://creativecommons.org/licenses/by-nc-sa/3.0/)
%
%%%%%%%%%%%%%%%%%%%%%%%%%%%%%%%%%%%%%%%%%

%----------------------------------------------------------------------------------------
%	PACKAGES AND THEMES
%----------------------------------------------------------------------------------------

\documentclass{beamer}

\mode<presentation> {

% The Beamer class comes with a number of default slide themes
% which change the colors and layouts of slides. Below this is a list
% of all the themes, uncomment each in turn to see what they look like.

%\usetheme{default}
%\usetheme{AnnArbor}
%\usetheme{Antibes}
%\usetheme{Bergen}
%\usetheme{Berkeley}
%\usetheme{Berlin}
%\usetheme{Boadilla}
%\usetheme{CambridgeUS}
%\usetheme{Copenhagen}
%\usetheme{Darmstadt}
%\usetheme{Dresden}
%\usetheme{Frankfurt}
%\usetheme{Goettingen}
%\usetheme{Hannover}
%\usetheme{Ilmenau}
%\usetheme{JuanLesPins}
%\usetheme{Luebeck}
\usetheme{Madrid}
%\usetheme{Malmoe}
%\usetheme{Marburg}
%\usetheme{Montpellier}
%\usetheme{PaloAlto}
%\usetheme{Pittsburgh}
%\usetheme{Rochester}
%\usetheme{Singapore}
%\usetheme{Szeged}
%\usetheme{Warsaw}

% As well as themes, the Beamer class has a number of color themes
% for any slide theme. Uncomment each of these in turn to see how it
% changes the colors of your current slide theme.

%\usecolortheme{albatross}
%\usecolortheme{beaver}
%\usecolortheme{beetle}
%\usecolortheme{crane}
%\usecolortheme{dolphin}
%\usecolortheme{dove}
%\usecolortheme{fly}
%\usecolortheme{lily}
%\usecolortheme{orchid}
%\usecolortheme{rose}
%\usecolortheme{seagull}
%\usecolortheme{seahorse}
%\usecolortheme{whale}
%\usecolortheme{wolverine}
\usepackage{mathrsfs}



%\setbeamertemplate{footline} % To remove the footer line in all slides uncomment this line
%\setbeamertemplate{footline}[page number] % To replace the footer line in all slides with a simple slide count uncomment this line

%\setbeamertemplate{navigation symbols}{} % To remove the navigation symbols from the bottom of all slides uncomment this line
}
\usepackage{CJKutf8}
\usepackage{graphicx} % Allows including images
\usepackage{booktabs} % Allows the use of \toprule, \midrule and \bottomrule in tables

% UTF-8 encoding
% CJKfonts package
% latex+dvips, latex+dvipdfm(x) or pdflatex


\begin{document}
\begin{CJK*}{UTF8}{bsmi}

%\documentclass{article}
%----------------------------------------------------------------------------------------
%	TITLE PAGE
%----------------------------------------------------------------------------------------

\title[Fuzzy Set Theory]{Fuzzy Set Theory and Its Applications} % The short title appears at the bottom of every slide, the full title is only on the title page

\author{鍾岳峰} % Your name
\institute[Yue-Fong,Chung] % Your institution as it will appear on the bottom of every slide, may be shorthand to save space
{
元智大學 \\ % Your institution for the title page
\medskip
\textit{pchomekimojuf@gmail.com} % Your email address
}
\date{\today} % Date, can be changed to a custom date



\begin{frame}
\titlepage % Print the title page as the first slide
\end{frame}

\begin{frame}
\frametitle{Overview} % Table of contents slide, comment this block out to remove it
\tableofcontents % Throughout your presentation, if you choose to use \section{} and \subsection{} commands, these will automatically be printed on this slide as an overview of your presentation
\end{frame}

%----------------------------------------------------------------------------------------
%	PRESENTATION SLIDES
%----------------------------------------------------------------------------------------

%------------------------------------------------
\section{Fuzzy Measures and Measures of Fuzziness} % Sections can be created in order to organize your presentation into discrete blocks, all sections and subsections are automatically printed in the table of contents as an overview of the talk
%------------------------------------------------

\subsection{4.1 Fuzzy Measures}% A subsection can be created just before a set of slides with a common theme to further break down your presentation into chunks

%------------------------------------------------

\begin{frame}
\frametitle{}

\begin{itemize}
\item In order to prevent confusion about fuzzy measures and measures of fuzziness, we shall first briefly describe the meaning and features of fuzzy measures. 
\item In the late 1970s, Sugeno defined a fuzzy measure as follows :\\
Sugeno [1977] :  \Large $\mathscr{B}$ ~ \normalsize is a Borel field of the arbitrary set (universe) X.

\end{itemize}

\end{frame}

%------------------------------------------------
%------------------------------------------------

\begin{frame}
\frametitle{}
A set function g defined on \Large $\mathscr{B}$ \normalsize that has the following properties is called a fuzzy measure:\\
\begin{block}{Definition 4-1}

~\\
1. g(0)=0,g(X)=1.\\
2. If A,B $\in$\Large $\mathscr{B}$\normalsize~ and A$ \subseteq $B,then g(A) $\leq$ g(B).\\
3. If $A_{n}~\in$ \Large $\mathscr{B}$\normalsize,$A_{1} \subseteq A_{2} \subseteq$...,then $ \lim_{n \rightarrow  \infty } $($A_{n}$)=g($ \lim_{n \rightarrow  \infty } A_{n}$).\\
~\\
\end{block}
Sugeno's measure differs from the classical measure essentially by relaxing the additivity property [Murofushi and Sugeno 1989, p. 201].\\
A different approach,however, is used by Klement and Schwyhla [1982].\\
The interested reader is referred to their article.
\end{frame}

%------------------------------------------------
%------------------------------------------------

\begin{frame}
\frametitle{}
Banon [1981] shows that very many measures with finite universe, such as probability measures, belief functions, plausibility measures, and so on, are fuzzy measures in the sense of Sugeno. \\
For this book, one measure-possibility-is of particular interest [see Dubois and Prade 1988a, p.7].\\
In the framework of fuzzy set theory, Zadeh introduced the notion of a possibility distribution and the concept of a possibility measure, which is a special type of the fuzzy measure proposed by Sugeno. \\
A possibility measure is defined as follows [Zadeh 1978; Higashi and Klir 1982]:

\end{frame}

%------------------------------------------------
%------------------------------------------------

\begin{frame}
\frametitle{}
Let P(X) be the power set of a set X.\\
A possibility measure is a function $ \Pi $: P(X) $ \mapsto $ [0, 1] with the properties\\
\begin{block}{Definition 4-2}
l. $ \Pi $(0)=0,$ \Pi $(X)=1\\
2. A $ \subseteq $ B $ \Rightarrow  \Pi   $(A) $\leq \Pi $(B)\\
3. $\Pi$ ($U_{i \in I}A_{i}$ ) = $sup_{i\in I} \Pi$ ($A_{i}$) with an index set I.
~\\
\end{block}
It can be uniquely determined by a possibility distribution function f: X $ \rightarrow $ [0, 1] by$\Pi$(A)=$sup_{x \in A}f$ f(x),A $\subset$ X.\\
It follows directly that f is defined by f(x) =$\Pi$(\{x\})$\forall_x \in X$ ~[Klir and Folger 1988, p. 122].\\
A possibility is not always a fuzzy measure [Puri and Ralescu 1982].\\
It is, however, a fuzzy measure if X is finite and if the possibility distribution is normal-that is, a mapping into [0, 1].
\end{frame}

%------------------------------------------------
%------------------------------------------------

\begin{frame}
\frametitle{}

\begin{block}{Example 4-1}
Let X = {0, 1, . . . , 10} .\\
$\Pi$(\{x\}) : = Possibility that x is close to 8.\\
~\\
\begin{table}[h]
\begin{tabular}{|l|l|l|l|l|l|l|l|l|l|l|l|}
\hline
x  & 0  & 1  & 2 & 3 & 4 & 5 & 6 & 7 & 8 & 9 & 10 \\ \hline
$\Pi$(\{x\}) & .0 & .0 & .0  & .0  & .0 & .1 & .5 & .8 & 1 & .8 & .5 \\ \hline
\end{tabular}
\end{table}
~\\
$\Pi$(A): = Possibility that A contains an integer close to 8.\\
A $\subset$~X$~ \Longrightarrow~$ $\Pi$(A) = $sup_{x \in A}$ $\Pi$(\{x\}) \\
For A = {2, 5, 9} we compute:\\
$\Pi$(A) = $sup_{x \in A}$ $\Pi$(\{x\}) \\
= sup\{$\Pi$(\{2\}), $\Pi$(\{5\}), $\Pi$(\{9\})\} \\
= sup\{0, .1, .8\}\\
=.8\\
~\\
~\\
\end{block}

\end{frame}

%------------------------------------------------

\subsection{4.2 Measures of Fuzziness} 

%------------------------------------------------

\begin{frame}
\frametitle{4.2}

\begin{itemize}
\item Measures of fuzziness, in contrast to fuzzy measures, try to indicate {\color{red}the degree of fuzziness of a fuzzy set.} 
\item A number of approaches to this end have become known. 
\item Some authors, strongly influenced by the \textbf{Shannon entropy} as a measure of information, and following de Luca and Termini [1972], consider a measure of fuzziness as {\color{red}a mapping \textbf{d} from the power set P(X) to [0, +\(\infty \)] that satisfies a number of conditions}.

\item Others [Kaufmann 1975] suggested an index of fuzziness as a \textbf{normalized distance}, and others [Yager 1979; Higashi and Klir 1982] base their concept of a measure of fuzziness on \textbf{the degree of distinction between the fuzzy set and its complement}.

\end{itemize}

\end{frame}

%------------------------------------------------

%------------------------------------------------

\begin{frame}
\frametitle{}
We shall, as an illustration, discuss two of those measures.\\
Suppose for both cases that \textbf{the support of A is finite}.\\
~\\
The first is as follows : Let $\mu_{\tilde{A}}(x)$ be the membership function of the fuzzy set $\tilde{A}$ for $x \in X$, X finite. It seems plausible that the measure of fuzziness d($\tilde{A}$) should then have the following properties [de Luca and Termini 1972]:\\
\begin{block}{}
\begin{itemize}
\item1. d($\tilde{A}$) = 0 if $\tilde{A}$ is a crisp set in X.
\item2. d($\tilde{A}$) assumes a unique maximum if $\mu_{\tilde{A}}(x)$ = $ \frac{1}{2} \forall x \in X$.
\item3. d($\tilde{A}) \geq  d(\tilde{A'}$) if $\tilde{A'}$ is "crisper" than $\tilde{A}$, i.e., if $\mu_{\tilde{A'}}(x) \leq$ $\mu_{\tilde{A}}(x)$ for $\mu_{\tilde{A}}(x) \leq \frac{1}{2}$ and  $\mu_{\tilde{A'}}(x) \geq$ $\mu_{\tilde{A}}(x)$ for $\mu_{\tilde{A}}(x) \geq \frac{1}{2}$. 
\item4. d(¢$\tilde{A}$) = d($\tilde{A}$) where ¢$\tilde{A}$ is the complement of $\tilde{A}$.
\end{itemize}
\end{block}
\end{frame}

%------------------------------------------------
%------------------------------------------------

\begin{frame}
\frametitle{}
De Luca and Termini suggested as a measure of fuzziness the {\color{red}"entropy"} of a fuzzy set [de Luca and Termini 1972, p. 305], which they defined as follows :\\
\begin{block}{Definition 4-3a}
The entropy as a measure of a fuzzy set $\tilde{A}$ = \{(x, $\mu_{\tilde{A}}(x))$\} is defined as\\
~\\
\centering\textbf{d($\tilde{A}$) = H($\tilde{A}$) + H(¢$\tilde{A}$), x $\in$ X}\\
\centering\textbf{H($\tilde{A}$) = - K$\sum_{i=1}^n \mu_{\tilde{A}}(x_{i})ln(\mu_{\tilde{A}}(x_{i}))$}\\
~\\
\end{block}
where \textbf{n} is the number of elements in the support of $\tilde{A}$ and K is a positive constant.
\end{frame}

%------------------------------------------------
%------------------------------------------------

\begin{frame}
\frametitle{}
Using \textbf{Shannon's function S(x)=-xlnx-(1-x)ln(1-x)}, de Luca and Termini simplify the expression in definition 4-3a to arrive at the following definition.\\
\begin{block}{Definition 4-3b}
The entropy d as a measure of fuzziness of a fuzzy set $\tilde{A}$ = \{x, $\mu_{\tilde{A}}(x)$\} is defined as\\
~\\
\centering\textbf{d($\tilde{A}$) = K$\sum_{i=1}^n S(\mu_{\tilde{A}}(x_{i}))$}\\
~\\
\end{block}
\end{frame}

%------------------------------------------------
%------------------------------------------------

\begin{frame}
\frametitle{}
\begin{block}{Example 4-2}
Let $\tilde{A}$= "integers close to 10" (see example 2-1d)\\
~~~~~~~~~~~~$\tilde{A}$ = \{(7, .1), (8, .5), (9, .8), (10, 1),(11, .8), (12, .5), (13,.1)\}\\
Let K = 1, so\\
~~~~~~~~~~~~d($\tilde{A}$) = .325+.693+.501+0+.501+.693+.611+.325=3.038\\
~\\
Furthermore, let $\tilde{B}$ ="integers quite close to 10"\\
~~~~~~~~~~~~$\tilde{B}$ = \{(6,.1),(7,.3),(8,.4),(9,.7),(10,1),(11,.8),(12,.5),(13,.3),(14,.l)\}\\
~~~~~~~~~~~~d($\tilde{B}$) = .325 +.611+.673+.611+0+.501+.693+.611+.325=4.35
\end{block}

\begin{block}{Shannon's function \textbf{S(x)=-xlnx-(1-x)ln(1-x)}}
\centering\textbf{$\mu_{\tilde{A}}(x)$=0.1,\\
-$\mu_{\tilde{A}}(x)$ln($\mu_{\tilde{A}}(x)$)-(1-$\mu_{\tilde{A}}(x)$)ln(1-$\mu_{\tilde{A}}(x)$)\\
=-0.1ln(0.1)-(1-0.1)ln(1-0.1)\\
$ \approx $0.325}\\
\end{block}
\end{frame}

%------------------------------------------------

%------------------------------------------------

\begin{frame}
\frametitle{}
The second measure is as follows: Knopfmacher [1975], Loo [1977], Gottwald [1979b], and others based their contributions on the Luca and Termini's suggestion in some respects.\\
~~\\
If $\tilde{A}$ is a fuzzy set in X and ¢$\tilde{A}$ is its complement, then in contrast to crisp sets, \textbf{it is not necessarily true} that\\
\begin{block}{}
~\\
~\\
\centering\textbf{$\tilde{A}~  \cup$  ¢$\tilde{A}$ = X}\\
\centering\textbf{$\tilde{A}~  \cap$~~¢$\tilde{A}$ = \(\phi\) }\\
~\\
\end{block}
This means that \textbf{fuzzy sets do not always satisfy the law of the excluded middle}, which is one of their major distinctions from traditional crisp sets. \\
~~\\
Some authors [Yager 1979; Higashi and Klir 1982] consider the relationship between $\tilde{A}$ and ¢$\tilde{A}$ to be the essence of fuzziness.
\end{frame}

%------------------------------------------------
%------------------------------------------------

\begin{frame}
\frametitle{}
Yager [1979] notes that the requirement of distinction between $\tilde{A}$ and ¢$\tilde{A}$ \textbf{is not satisfied by fuzzy sets}.
\\
He therefore suggests that any measure of fuzziness \textbf{should be a measure of the lack of distnction between $\tilde{A}$ and ¢$\tilde{A}$ or $\mu_{\tilde{A}}(x)~ and~ \mu_{~~~¢\tilde{A}}(x)$}.\\
As a possible metric to measure the distance between a fuzzy set and its complement, Yager suggests:\\
\begin{block}{Definition 4-4}
~\\
~\\
\centering\textbf{\(D_{p}( \tilde{A},~~¢\tilde{A}) = [~ \sum_{i=1}^n   | \mu_{\tilde{A}}( x_{i}) - \mu_{~~~¢\tilde{A}}( x_{i} ) |^{p} ~]^ \frac{1}{p}  ~~~~p=1,2,3...\)}\\
~\\
\raggedright Let S ~=~supp($\tilde{A}$): $D_{p}$(S, ¢S) =~$\|S\|^\frac{1}{p}$\\
~\\
\end{block}

\end{frame}

%------------------------------------------------
%------------------------------------------------

\begin{frame}
\frametitle{}

\begin{block}{Definition 4-5[Yager 1979](1/2)}
~\\
~\\
A measure of the fuzziness of $\tilde{A}$ can be defined as\\
~\\
\centering\textbf{$f_{p}(\tilde{A})$ = 1 - $\frac{D_{p}(\tilde{A},~~~¢\tilde{A})}{\|supp(\tilde{A})\|}$}\\
~\\
\raggedright So $f_{p}$($\tilde{A}$) $\in$ [0, 1]. This measure also satisfies properties 1 to 4 required by de Luca and Termini (see above).\\
~\\
~\\
\end{block}

\end{frame}

%------------------------------------------------
%------------------------------------------------

\begin{frame}
\frametitle{}

\begin{block}{Definition 4-5[Yager 1979](2/2)}
~\\
~\\
For p = 1, $D_{p}$($\tilde{A}$, ¢$\tilde{A}$) yields \textbf{the Hamming metric}\\
~\\
\centering\textbf{$D_{1}(\tilde{A},~~¢\tilde{A})$ = $\sum_{i=1}^n |\mu_{\tilde{A}}( x_{i})- \mu_{~~~¢\tilde{A}}( x_{i} )| $}\\
~\\
\raggedright Because $\mu_{\textbf{¢}\tilde{A}}( x) = 1 - \mu_{\tilde{A}}( x)$, this becomes\\
~\\
\centering\textbf{$D_{1}(\tilde{A},~~¢\tilde{A})$ = $\sum_{i=1}^n |2\mu_{\tilde{A}}( x_{i})- 1| $}\\
~\\
\raggedright For p = 2, we arrive at \textbf{the Euclidean metric}\\
~\\
\centering\textbf{$D_{2}(\tilde{A},~~¢\tilde{A})$ = $(\sum_{i=1}^n (\mu_{\tilde{A}}( x_{i})- \mu_{~~~¢\tilde{A}}( x_{i} ))^2 )^ \frac{1}{2} $}\\
~\\
\raggedright and for $\mu_{\textbf{¢}\tilde{A}}( x) = 1 - \mu_{\tilde{A}}( x)$, we have\\
~\\
\centering\textbf{$D_{2}(\tilde{A},~~¢\tilde{A})$ = $(\sum_{i=1}^n (2\mu_{\tilde{A}}( x_{i})- 1 ))^2 )^ \frac{1}{2} $}\\
~\\
\end{block}

\end{frame}

%------------------------------------------------
%------------------------------------------------

\begin{frame}
\frametitle{}

\begin{block}{Example 4-3(1/2)}
~\\
Let $\tilde{A}$ = "integers close to 10" and\\
$\tilde{B}$ = "integers quite close to 10" be defined as in {\color{red}example 4-2.}\\
Applying the above derived formula, we compute for p = 1:\\
~\\
\centering\textbf{$D_{1}(\tilde{A},~~¢\tilde{A})$ =.8+0+.6+1+.6+0+.8}\\
\textbf{=3.8}\\
\centering\textbf{$\|supp(\tilde{A})\| = 7$}\\
\raggedright so $f_{1}(\tilde{A}) = 1 - \frac{3.8}{7}$ = {\color{red}0.457.}\\
~\\
Analogously,\\
\centering $D_{2}(\tilde{B},\textbf{¢}\tilde{B})$ = 4.6\\
\centering\textbf{$\|supp(\tilde{B})\| = 9$}\\
\raggedright so $f_{1}(\tilde{B}) = 1 - \frac{4.6}{9}$ = {\color{red}0.489.}\\
~\\
~\\
\end{block}

\end{frame}

%------------------------------------------------
%------------------------------------------------

\begin{frame}
\frametitle{}

\begin{block}{Example 4-3(2/2)}
~\\
~\\
Similarly, for p = 2, we obtain\\
\centering $D_{2}(\tilde{A},\textbf{¢}\tilde{A})$ = 1.73\\
\centering\textbf{$\|supp(\tilde{A})\| = 2.65$}\\
\raggedright so $f_{2}(\tilde{A}) = 1 - \frac{1.73}{2.65}$ = {\color{red}0.347} ,and\\
~\\
\centering $D_{2}(\tilde{B},\textbf{¢}\tilde{B})$ = 1.78\\
\centering\textbf{$\|supp(\tilde{B})\| = 1$}\\
\raggedright so $f_{2}(\tilde{B}) = 1 - \frac{1.78}{3}$ = {\color{red}0.407.}\\
~\\
~\\
\end{block}

\end{frame}

%------------------------------------------------
%------------------------------------------------

\begin{frame}
\frametitle{}

\begin{itemize}
\item The reader should realize that {\color{red}the complement of a fuzzy set is not uniquely defined} [see Bellman and Giertz 1973; Dubois and Prade 1982a; Lowen 1978]. 
\item It is therefore not surprising that for other definitions of the complement and for other measures of distance, other measures of fuzziness will result, even though they all focus on the distinction between a fuzzy set and its complement [see, for example, Klir 1987, p. 141]. 
\item Those variations, as well as {\color{red}extension of measures of fuzziness to nonfinite supports, will not be considered here}; neither will the approaches that define fuzzy measures of fuzzy sets [Yager 1979].

\end{itemize}

\end{frame}

%------------------------------------------------
%------------------------------------------------
\begin{frame}
\Huge{\centerline{Thank you for your attention}}
\end{frame}

%----------------------------------------------------------------------------------------

\clearpage\end{CJK*}
\end{document}